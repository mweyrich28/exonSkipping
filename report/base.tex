\documentclass[12pt]{article}
\usepackage{paper,math}
\usepackage{biblatex}
\addbibresource{references.bib}

\title{Genomorientierte Bioinformatik - Report ExonSkipping}
\author{
  Malte Weyrich
}
\date{\today}

\hypersetup{
  pdftitle = {Example Paper Title},
  pdfauthor = {Author One, Author Two},
}

% Conditionally display thoughts (hide by switching to `\boolfalse`)
\boolfalse{INCLUDECOMMENTS}

\begin{document}

% Title Page -------------------------------------------------------------------
\maketitle
\begin{abstract}
	\textbf{Exon Skipping Splicing Events} (\textit{ES-SE}) beschreiben, wie co- oder posttranslational,
	manche Exons eines Transkripts durch das Splei\ss osom herausgeschnitten oder übersprungen werden, während
	in anderen Transkripten des selben Gens, diese weiterhin Teil der mRNA bleiben.
	Die \textit{ES-SE} lassen sich anhand von \textbf{Gene Transfer Format} (\textit{GTF}) files,
	also Genom Annotations Dateien ablesen und analysieren.
	Im Folgenden wird ein Programm zur Erkennung von allen \textit{ES-SE} innerhalb eines Genoms
	anhand seiner Logik, Laufzeit und Ergebnisse analysiert, wobei nur \textit{ES-SE} berücksichtigt werden,
	die protein-kodierende Transkripte betreffen. Das Programm wurde auf allen verfügbaren
	\textit{GTF} Dateien in \textit{/mnt/biosoft/praktikum/genprakt/gtfs/} ausgeführt.
	Der Source Code und alle dazugehörigen Komponenten sind auf \href{https://github.com/mweyrich28/exonSkipping}{GitHub} zu finden.
\end{abstract}

\newpage
\tableofcontents
\newpage


% Paper ------------------------------------------------------------------------

% ------------------------------------------------------------------------------
\section{ES-SE Definition}\label{sec:problem}
In einem Gen kann jedes Transkript jeweils mehrere \textit{ES-SE} haben.
Ein \textit{ES-SE} involviert immer jeweils mindestens eine \textbf{Splice Variant} (\textit{SV}) und einen
\textbf{Wild Type} (\textit{WT}). Beide dieser Begriffe beziehen sich auf Transkripte eines Gens $G$.
Ein \textit{SV} ist ein Transkript $T_{SV}$, welches ein Intron $I$ mit Startposition $I_{S}$ und Endposition
$I_{E}$ besitzt, was gleichzeitig bedeutet, dass es in $T_{SV}$ zwei Exons $A, B$ gibt, die $I$ flankieren. Somit endet $A$ bei $I_{S} - 1 = A_{E}$ und $B$ startet bei $I_{E} + 1 = B_{S}$.
Zudem ist die Position $B_{pos} - A_{pos} = 1$, wobei sich $A_{pos}$ auf die Position von Exon $A$ relativ gesehen
zu allen anderen Exons von $T_{SV}$ bezieht.
Ein \textit{WT} wäre nun ein weiteres Transkript $T_{WT}$ des selben Gens $G$, welches ebenfalls
zwei Exons $C, D$ besitzt mit $C_{pos} < D_{pos}$, wobei $C_{E} = I_{S} - 1$ und $D_{S} = I_{E} + 1$,
jedoch gilt für $C, D$: $D_{pos} - C_{pos} > 1$.
Dies bedeutet, dass die Exons von $T_{WT}$ zwischen $C$ und $D$ in $T_{SV}$ übersprungen wurden.
Es kann pro Event mehrere \textit{SV}'s und \textit{WT}'s geben.

\section{Java Programm}
\subsection{Logik}\label{sec:logik}
Der Workflow der \textit{JAR} lässt sich in drei Schritte Aufteilen:
\begin{enumerate}
	\item[(A)] \textbf{Einlesen der \textit{GTF} Datei und Initialisierung der Datenstruktur}

		Zum einlesen wird die \textit{GTF} Datei zuerst nach relevanten Zeilen gefiltert, denn
		für uns sind momentan nur Zeilen relevant, die in der 3. Spalte entweder
		\textit{"exon"} oder \textit{"CDS"} stehen haben. Hierbei wird vermieden,
		die Methode \textit{String.split("\textbackslash t")} zu verwenden.
		Stattdessen wird in einem \textit{for loop} jedes Zeichen einzeln betrachtet.
		Dabei werden Zeilen die mit einem \textit{"\#"} anfangen direkt übersprungen.
		Für alle anderen Zeilen werden die Anzahl der \textit{tabs} gezählt und nach
		dem zweiten \textit{tab}, werden alle darauf folgenden Zeichen zu einem
		\textit{String} zusammen konkateniert, bis der dritte \textit{tab}
		erreicht wurde. Falls der entstandenen String $\in \left\{\text{"exon"}, \text{"CDS"} \right\}$, wird die
		Zeile einer \textit{ArrayList<String>} hinzugefügt, ansonsten wird mit der nächsten Zeile weiter gemacht.
		Diese Liste enthält am Ende alle relevanten Zeilen.
		Jede der relevanten Zeilen werden nun mit \textit{String.split("\textbackslash t")} in
		ein \textit{String[] mainComponents} geschrieben. Die \textit{attributeColumn} wird aus
		\textit{mainComponents} extrahiert (auch als ein \textit{String[]} Names \textit{attributes}),
		indem man \textit{mainComponents[mainComponents.length - 1]} am \textit{";"} teilt.

		Mit diesen zwei Komponenten pro Zeile wird als erstes die \textit{gene\_id} abgespeichert
		und überprüft, ob wir eine neue \textit{gene\_id} erreicht haben.
		Falls ja, wird ein neues Gen erstellt. Für die darauf folgenden Zeilen wird überprüft,
		ob wir ein neues Transkript erreicht haben. Neue Transkripte werden in einer
		\textit{ArrayList<Transkript>} des dazugehörigen Gens abgespeichert.
		Transkripte wiederum besitzen eine \textit{ArrayList<CodingDnaSequence> cdsList} und zwei
		\textit{HashMap<Integer, CodingDnaSequence>} \textit{cdsStartIndices}, \textit{cdsEndIndices}. Die Transkripte werden mit den dazugehörigen
		\textit{CodingDnaSequence}'s befüllt, wobei für jede erstellte \textit{CodingDnaSequence} die Start- und
		Endposition in den jeweiligen \textit{HashMap}'s als Key auf das erstellte Objekt verweisen.
		Zudem wird mit einer Zählvariable \textit{int cdsCount} die Position der \textit{CodingDnaSequence}'s
		innerhalb des Transkripts in dem \textit{CodingDnaSequence} Objekt gespeichert.

	\item[(B)] \textbf{Generieren der \textit{ES-SE}}

		Zum generieren der \textit{ES-SE} werden als erstes für alle in dem Genom abgespeicherten
		Gene, die dazugehörigen \textit{Introns} errechnet und in einem \textit{HashSet<Introns>}
		innerhalb des Gens abgespeichert. Dafür werden alle Transkripte eines Gens und deren
		\textit{CodingDnaSequence}'s angeschaut. Die Introns werden dann mit jeweils
		zwei \textit{CodingDnaSequence}'s berechnet
		(bei Genen die sich auf dem \textit{"-"} Strang befinden, müssen zuerst die
		\textit{cdsList}'s aller Transkripte invertiert werden und die
		Positionen der \textit{CodingDnaSequence}'s neu berechnet werden.
		Das ist später relevant für die Identifikation der \textit{WT}'s):
		\begin{verbatim}

// invert cdsList of transcripts
public void invertTranscripts() {
    for (int i = 0; i < transcripts.size(); i++) {
        Transcript currTranscript = transcripts.get(i);
        currTranscript.reversCdsList();
        // updating pos attribute of each cds
        for (int j = 0; j < currTranscript.getCdsList().size(); j++) {
            currTranscript.getCdsList().get(j).setPos(j);
        }
    }
}

// generating introns
for (Transcript transcript : transcripts) {
    for (int i = 0; i < transcript.getCdsList().size() - 1; i++) {
        int intronStart = transcript.getCdsList().get(i).getEnd() + 1;
        int intronEnd = transcript.getCdsList().get(i + 1).getStart() - 1;
        Intron intron = new Intron(intronStart, intronEnd);
        introns.add(intron);
    }
}
    \end{verbatim}

		Anschlie\ss end wird für jedes Gen $G$ über die Intron Liste iteriert. Für jedes Intron $I$
		müssen alle Transkripte von $G$ nach \textit{CodingDnaSequence}'s $A, B$ durchsucht werden, die
		die Bedingung $A_{E} + 1 = I_{S}$ und $B_{S} - 1 = I_{E}$. Dies kann mit Hilfe der zwei
		\textit{HashMap<Integer, CodingDnaSequence>} Objekte durchgeführt werden.
		Zudem wir für jedes Intron $I$ jeweils 4 leere \textit{HashSet<String>}'s erstellt:
		\begin{enumerate}
			\item[1.] \textit{SV\_INTORN}: enthält \textit{"intronStart:intronEnd"} des momentanen Introns $I$
			\item[2.] \textit{SV\_PROTS}: enthält die \textit{proteinId} von \textit{CodingDnaSequence} $A$
			\item[3.] \textit{WT\_INTORN}: enthält alle \textit{"intronStart:intronEnd"} Koordinaten, die zwischen $A$ und $B$ liegen
			\item[4.] \textit{WT\_PROTS}: enthält die \textit{proteinId}'s von allen \textit{CodingDnaSequence}'s die zwischen $A$ und $B$ liegen
		\end{enumerate}
		Nun gibt es zwei Möglichkeiten:
		\begin{enumerate}
			\item  $A_{E} + 1 = I_{S}$ und $B_{S} - 1 = I_{E}$ und $B_{pos} - A_{pos} = 1$
			\item  $A_{E} + 1 = I_{S}$ und $B_{S} - 1 = I_{E}$ und $B_{pos} - A_{pos} > 1$
		\end{enumerate}

		Falls $i$ eintrifft, handelt es sich um ein \textit{SV} und es wird die \textit{proteinId} von \textit{CodingDnaSequence} $A$
		in \textit{SV\_PROTS} aufgenommen.
		Ansonsten werden bei Fall $ii$ alle \textit{proteinId}'s der \textit{CodingDnaSequence}'s zwischen $A$ und $B$ zu
		\textit{WT\_PROTS} und alle Introns zwischen $A$ und $B$ zu \textit{WT\_INTORN} hinzugefügt.
		Dabei werden ebenfalls Werte wie \textit{min/max\_skipped\_exon/bases} berechnet:
		\begin{verbatim}
// add all introns of WT to WT_INTRON and all cdsids/prot_ids to WT_prots
int skippedBases = 0;
for (int i = cdsFront.getPos() ; i < cdsBehind.getPos(); i++) {
    int wtIntronStart = cdsList.get(i).getEnd() + 1;
    int wtIntronEnd = cdsList.get(i+1).getStart();
    WT_INTRON.add(wtIntronStart + ":" + wtIntronEnd);

    // like this i add many ids twice but that's fine :)
    WT_PROTS.add(cdsFront.getId());
    WT_PROTS.add(cdsBehind.getId());

    if (i > cdsFront.getPos() && i < cdsBehind.getPos()) {
        // we are in a cds that was skipped
        // → get end - start + 1 = length → add to skipped bases
        skippedBases += cdsList.get(i).getEnd() 
                        - cdsList.get(i).getStart() + 1;
    }
}
    \end{verbatim}
		Ein \textit{ES-SE} wird nur in die \textit{ArrayList<String> events} aufgenommen, falls es für das momentane
		Intron $I$ mindestens einen \textit{WT} gab.

	\item[(C)] \textbf{Erstellen der \textit{<out>.tsv} Datei}

		Die \textit{ArrayList<String> events} enthält nun alle \textit{ES-SE} als \textit{String} in bereits
		korrekter Formatierung. In einem \textit{for loop} wird die Lösung Zeile für Zeile in ein \textit{out.tsv}
		geschrieben.
\end{enumerate}

\subsection{Korrektheit}
In der Einleseroutine werden alle relevanten Zeilen verarbeitet und das Genom
korrekt Initialisiert, sofern die Struktur von dem \textit{GTF} den \href{https://asia.ensembl.org/info/website/upload/gff.html}{offiziellen Konventionen} folgt und die jeweiligen
\textit{CodingDnaSequence}'s in korrekter Reihenfolge (je nach \textit{"-"/"+"} Strang) vorliegen.
Zudem werden, falls es keine \textit{"protein\_id"} für eine gegebene Zeile gibt, nach der
\textit{"ccdsid"} gesucht und falls es diese nicht gibt, wird die \textit{"protein\_id"}
mit \textit{"NaN"} überschrieben. So werden alle Zeilen, die \textit{"CDS"} in ihrer
dritten Spalte stehen haben, genutzt, um das Genom aufzufüllen.
Für das Errechnen der \textit{ES-SE} in Schritt (B) gilt folgendes:
Alle möglichen Introns in einem Gen werden überprüft und für jedes Intron
werden alle Transkripte des jeweiligen Gens auf bei $I_{S} - 1$ endende  und
bei $I_{E} + 1$ startende \textit{CodingDnaSequence}'s $A, B$ abgefragt. Für
jeden \textit{SV} oder \textit{WT} Kandidaten wird anschlie\ss end geschaut,
ob es zwischen $A$ und $B$ weitere \textit{CodingDnaSequence}'s gibt und
je nach dem ein \textit{ES-SE} entdeckt oder nicht.
So ist das Programm unter der Annahme, dass die \textit{GTF} Datei fehlerfrei ist, korrekt.



\subsection{Laufzeit}
Die Laufzeitanalyse wird in die drei Segmente aus \ref{sec:logik} unterteilt.
\begin{enumerate}
	\item[(A)] \textbf{Einlesen der \textit{GTF} Datei und Initialisierung der Datenstruktur}

		Für eine \textit{GTF} Datei mit $m$ Zeilen benötigt die Selektion der relevanten Zeilen
		schon mal mindestens $m$ Vergleiche, da jede Zeile überprüft werden muss.
		Für jede Zeile wird ein Substring ab dem zweiten \textit{Tab} bis zum dritten \textit{Tab} erstellt (au\ss er bei Kommentaren, diese werden übersprungen).
		Die Anzahl der Vergleiche pro Zeile ist kleiner als die Länge der Zeile (da wir ab dem dritten \textit{Tab} abbrechen) und lässt
		sich als $a < m.length$ beschreiben. Also
		\begin{equation}
			\mathcal{O}(m \cdot a) \implies \mathcal{O}(m)
		\end{equation}
		,da $a$ eine Konstante ist.

		Bei einer \textit{GTF} Datei mit $m$ validen Zeilen (d.h. jede Zeile hat entweder einen \textit{"exon"} oder \textit{"CDS"} Eintrag) bleiben
		nach dem Filtern $m$ Zeilen übrig.
		Für jede dieser Zeilen muss:
		\begin{enumerate}
			%% https://softwareengineering.stackexchange.com/questions/331909/whats-the-complexity-of-javas-string-split-function
			\item Die Zeile am \textit{Tab} geteilt werden:
			      \begin{equation}
				      \mathcal{O}(n)
			      \end{equation}
			      , wobei $n$ die Länge der Zeile ist.
			\item Die letzte Komponente aus i. am \textit{;} geteilt werden: lässt sich ebenfalls mit
			      \begin{equation}
				      \mathcal{O}(n)
			      \end{equation}
			      von oben beschränken.
			\item Die \textit{gene\_id} aus den Attributen aus ii. mit \textit{String parseAttributes(String[] attributeEntries, String attributeName)}
			      abfragen, also:
			      \begin{equation}
				      \mathcal{O}(e \cdot e_{l})
			      \end{equation}
			      , wobei $e$ die Länge des \textit{attributeEntries} Arrays ist und
			      $e_{l}$ die Länge des längsten Eintrags in \textit{attributeEntries} ist, da wir im Worst Case
			      über alle Einträge in \textit{attributeEntries} iterieren ($e$) und für jeweils jeden Eintrag
			      mindesten vier String Operationen (\textit{trim()}, \textit{indexOf()}, zwei mal \textit{substring()})
			      und eine Vergleichsoperation (\textit{equals()}) aufrufen, welche maximal $e_{l}$ viele
			      Operationen benötigen. Also $\mathcal{O}(e \cdot 5 \cdot e_{l}) = \mathcal{O}(e \cdot e_{l})$.
			      Für jedes weitere Vorkommen von \textit{parseAttributes()} wird diese Komplexität angenommen.
			\item Ein neues Gen initialisiert werden und der \textit{gene\_name} abgefragt werden (falls ein neues Gen erreicht wurde), also:
			      \begin{equation}
				      \mathcal{O}(1 + e \cdot e_{l})
			      \end{equation}
			\item Die \textit{transcript\_id} abgefragt werden, also
			      \begin{equation}
				      \mathcal{O}(e \cdot e_{l})
			      \end{equation}
			\item Falls es sich um einen \textit{"CDS"} Eintrag handelt, muss die \textit{protein\_id} abgefragt werden, also:
			      \begin{equation}
				      \mathcal{O}(e \cdot e_{l})
			      \end{equation}
			      und in Konstanter Zeit ggf. neue Objekte erstellt, oder auf bereits existierende Objekte
			      zugreifen, um ein neues \textit{CodingDnaSequence} Objekt zu erstellen.
		\end{enumerate}

		Insgesamt hat die Einleseroutine also eine Komplexität von
		\[
			(A) \hspace{1em} \mathcal{O}(m \cdot 2 \cdot n \cdot 4 \cdot  (e \cdot e_{l})) = \mathcal{O}(m \cdot n \cdot e \cdot e_{l}) \in  \mathcal{O}(m^{2})
			.\]
		,da $m > n > e_{l} >= e$ und somit $\mathcal{O}(m^{2})$ eine valide obere Schranke darstellt.
	\item[(B)] \textbf{Generieren der \textit{ES-SE}}

		Sei $g$ die Anzahl an Genen in unserem Genom. Da wir vom Worst Case ausgehen sagen wir,
		dass sich jedes Gen auf dem \textit{"-"} Strang befindet.
		So muss zuerst in jedem Transkript jedes Gens die \textit{cdsList} invertiert werden.
		Dies geschieht in:
		\begin{equation}
			\mathcal{O}(g \cdot g_{t} \cdot 2 \cdot t_{c}) = \mathcal{O}(g \cdot g_{t} \cdot t_{c})
		\end{equation}
		, wobei $g_{t}$ die grö\ss te Anzahl an Transkripten von $g$ ist und $t_{c}$ die längste \textit{cdsList} eines Transkripts ist.
		Die Konstante $2$ kommt zustande, da zuerst die \textit{cdsList} mit \textit{Collections.reverse()} umgekehrt wird ($\mathcal{O}(g_{t})$) und
		dann nochmals in $\mathcal{O}(g_{t})$ durchlaufen wird, um die in den \textit{CodingDnaSequence} Objekten gespeicherte Position
		anzupassen.
		Dann werden für jedes Gen $g$ die Introns generiert. Dies geschieht ebenfalls in
		\begin{equation}
			\mathcal{O}(g \cdot g_{t} \cdot t_{c})
		\end{equation}
		Die \textit{ES-SE} werden in der \textit{getEvents()} Methode berechnet.
		Dafür müssen für alle Gene alle Introns und alle Transkripte überprüft werden.
		Also schon mal $\mathcal{O}(g \cdot g_{i} \cdot g_{t})$.
		Hier ist $g_{i}$ die grö\ss te Anzahl an Introns von allen Genen und $g_{t}$ wieder
		die grö\ss te Anzahl an Transkripten aller Gene. Alle anderen Operationen sind
		konstant in ihrer Komplexität, da bei ihnen lediglich bereits existente Werte
		in \textit{HashMaps} oder Objekten abgefragt werden.
		Nur falls ein \textit{WT} entdeckt wird, wird in einem \textit{for loop} über die
		\textit{CodingDnaSequence}'s zwischen $C$ und $D$ (siehe \ref{sec:problem}. ES-SE Definition) iteriert.
		Sei $mSE$ (= maxSkippedExons) also die von allen \textit{ES-SE} eines Genoms
		maximale Anzahl an übersprungenen Exons, so wäre die gesamte Komplexität von
		(B):
		\begin{equation}
			(B) \hspace{1em} \mathcal{O}(\overbrace{2 \cdot (g \cdot g_{t} \cdot t_{c})}^{\text{(8) \& (9)}}  +\underbrace{g \cdot g_{i} \cdot g_{t} \cdot mSE}_{\text{getEvents()}} ) = \mathcal{O}(g \cdot g_{t} \cdot t_{c} + g \cdot g_{i} \cdot g_{t} \cdot mSE)
		\end{equation}

	\item[(C)] \textbf{Erstellen der \textit{<out>.tsv} Datei}

		Hat eine Komplexität von
		\begin{equation}
			(C) \hspace{1em} \mathcal{O}(E)
		\end{equation}
		, wenn $E$ die Menge aller \textit{ES-SE} ist.
\end{enumerate}

Zusammenfassend also eine Gesamtkomplexität von:
\begin{equation}
	(A) + (B) + (C) = \mathcal{O}( m^{2} + g \cdot g_{t} \cdot t_{c} + g \cdot g_{i} \cdot g_{t} \cdot mSE + E) \in \mathcal{O}(m^{2})
\end{equation}
, da $m^{2} > g \cdot g_{t} \cdot t_{c}$ und $m^{2} > g \cdot g_{i} \cdot g_{t} \cdot mSE + E$.
Das bedeutet, dass die Kosten der Gesamtoperation im Wesentlichen durch das Einlesen und Strukturieren der Daten (Teil A) dominiert werden,
wenn man davon ausgeht, dass $m$ in der Praxis größer als $g$, $g_{t}$, $t_{c}$, $g_{i}$ und $mSE$ ist.

\subsection{Benchmarking}
Für das Benchmarking wird jeweils \textit{/mnt/biosoft/praktikum/genprakt/gtfs/Homo\_sapiens.GRCh38.86.gtf} verwendet, da sie die grö\ss te \textit{GTF} Datei mit $1.4GB$ ist.
Die zwei \textit{JARs} ($M$ und $L$) aus unserer Gruppe wurden jeweils 30 Mal ausgeführt und davon dann ein Durchschnitt errechnet.

\begin{figure}[htpb]
	\centering
	\includegraphics[width=0.8\textwidth]{./plots/benchmark_time.jpg}
	\caption{Methoden Durchschnittslaufzeit der Schritte A, B, C  in ms nach 30 facher Ausführung auf Hardware: \textit{AMD Ryzen 7 PRO 4750U with Radeon Graphics (16) @ 1.700GHz}}

	\label{fig:-plots-benchmark_time-jpg}
\end{figure}

In \ref{fig:-plots-benchmark_time-jpg} ist eindeutig zu sehen, wie das Einlesen und Initialisieren
der Datenstruktur aus Schritt (A), die Dominante Komponente beider \textit{JARs}, mit $6721ms$ und $26977ms$ bildet, während die Generierung der
\textit{ES-SE} lediglich $262ms$ und $498ms$ benötigt.
Für die \textit{Memory Allocations} wurde der in \textit{IntelliJ} zur Verfügung gestellter \textit{Profiler} verwendet.
Wichtig ist hierbei der Unterschied zwischen \textit{RAM} und \textit{Memory Allocations}:
Der \textit{Profiler} erfasst, wie viel Speicher ein Programm insgesamt anfordert, 
auch wenn ein Großteil davon später wieder durch den \textit{Garbage Collector} freigegeben wird.
Der \textit{Profiler} gibt also eher eine Obergrenze des Speicherverbrauchs an, nicht den exakten \textit{RAM}-Verbrauch.
Der Schritt (A) ist auch hier dominant und fordert in \textit{JAR} $M$ insgesamt $10.33GB$ an Speicher an.
Von diesen $10.33GB$ werden alleine $6.72GB$ von der Methode \textit{String.split()} gefordert.
\textit{JAR} $L$ hingegen benötigt $65.32GB$ an Memory Allocations, wobei der Hauptteil 
dieses Volumens (ca. $60GB$) für die Methode \textit{String.replaceAll()} aufgewendet wird.
Für die Berechnung der \textit{ES-SE} werden lediglich $288.11MB$ ($M$) und $296.91MB$ ($L$) in Anspruch genommen.
Die starken Diskrepanzen in Schritt (A) liegen daran, dass die \textit{JAR} $L$ alle Zeilen mit dem 
\textit{Parser} einlie\ss t und von diesen jedes mal die \textit{attributes} vollständig in einer
Datenstruktur abspeichert. Die Methode, die die \textit{attributes} in $L$ verarbeitet, ruft für jedes Attribut-Paar
(\textit{key}, \textit{val}), die Methode \textit{String.replaceAll("\textbackslash "")} auf, welche teuer 
in Laufzeit und \textit{Memory Allocations} ist.
Da es in \textit{/mnt/biosoft/praktikum/genprakt/gtfs/Homo\_sapiens.GRCh38.86.gtf} insgesamt 2575498 Einträge gibt
und pro Entrag jeweils mehrere Attribute, wird \textit{String.replaceAll()} extrem häufig aufgerufen.
Zusätzlich werden die Attibute in \textit{HashMap}'s gespeichert, was ein zusätzlicher Faktor ist.
In \textit{JAR} $M$ werden die Zeilen wie in \ref{sec:logik} beschreiben nur dann tatsächlich bearbeitet,
wenn sie einen \textit{"exon"} oder \textit{"CDS"} Eintrag beinhalten, zudem werden die \textit{attributes} effizienter 
verarbeitet. Somit ist der \textit{Parser} aus $M$ zwar schneller, jedoch weniger versatil für andere Probleminstanzen.


\section{Ergebnisse}\label{sec:res}
Für die Analyse wurde die \textit{GTF} Datei \textit{/mnt/biosoft/praktikum/genprakt/gtfs/Saccharomyces\_cerevisiae.R64-1-1.75.gtf} ausgelassen,
da es hier zu keinen \textit{ES-SE} gekommen ist. Die Ursache dafür ist wahrscheinlich, dass
es, obwohl es in der Hefe auch zum Splei\ss en kommt, in der gegebenen \textit{GTF} Datei keine protein-kodierenden Transkripte mit
Splicing gab, oder allgemein keine \textit{ES-SE} aufgetreten sind. 
In der Hefe gibt es insgesamt sehr wenige Introns (ca. 300), verglichen mit ($> 140.000$) in dem Menschen (\cite[p.~1525]{doi:10.1073/pnas.0610354104}),
was die Abwesenheit von \textit{ES-SE} in \textit{Saccharomyces\_cerevisiae.R64-1-1.75.gtf} zusätzlich erklärt. 
Die \textit{GTF} Dateien und die dazugehörigen Ergebnisse werden ab jetzt mit den folgenden IDs aus Tabelle 
\ref{tab:label} bezeichnet:

\begin{table}[htpb]
	\centering
	\caption{Liste der verwendeten \textit{GTF} Dateien}
	\label{tab:label}
	\begin{tabular}{l|l}
		\textbf{ID} & \textbf{\textit{GTF} Datei} \\ \hline
		h.ens.67    & Homo\_sapiens.GRCh37.67.gtf \\
		h.ens.75    & Homo\_sapiens.GRCh37.75.gtf \\
		h.ens.86    & Homo\_sapiens.GRCh38.86.gtf \\
		h.ens.90    & Homo\_sapiens.GRCh38.90.gtf \\
		h.ens.93    & Homo\_sapiens.GRCh38.93.gtf \\
		m.ens.75    & Mus\_musculus.GRCm38.75.gtf \\
		h.gc.10     & gencode.v10.annotation.gtf  \\
		h.gc.25     & gencode.v25.annotation.gtf  \\
	\end{tabular}
\end{table}


Als erstes wird die Anzahl an aller protein-kodierenden Gene einer \textit{GTF} mit der Anzahl an Genen mit \textit{ES-SE} und der Gesamtanzahl an \textit{ES-SE}
verglichen:

\begin{figure}[htpb]
	\centering
	\includegraphics[width=0.8\textwidth]{./plots/genes.jpg}
	\caption{Vergleich zwischen Gene Gesamt, Gene mit \textit{ES-SE} und \textit{ES-SE} Gesamt pro \textit{GTF}}
	\label{fig:-plots-genes-jpg}
\end{figure}

In Abbildung \ref{fig:-plots-genes-jpg} sieht man, wie die Anzahl an Genen in den verschiedenen \textit{GTF} Dateien des
Menschen in einem Intervall von $[20320; 23393]$ variieren. Dies liegt daran, dass die
\textit{GTF} Dateien jeweils von unterschiedlichen Assemblies und Annotations Versionen stammen,
welche beide einen starken Einfluss auf die resultierende \textit{GTF} haben (und somit auch auf das Ergebnis der \textit{JAR}).
Wie zu erwarten, hat bei den zum Menschen zugehörigen \textit{GTF} Dateien, die mit den meisten Genen auch die meisten \textit{ES-SE}.
Die Dateien \textit{"h.gc.25"}, \textit{"h.ens.67"}, \textit{"h.ens.86"}, \textit{"h.ens.90"}, \textit{"h.ens.93"} haben alle eine sehr
ähnliche Verteilung der \textit{"ES-SE Gesamt"} und \textit{"Gene mit ES-SE"} Kategorie, obwohl die \textit{GTF} Datei von
\textit{"h.ens.67"} mehr Gene beinhaltet, als die drei anderen \textit{GTF} Dateien.
\textit{"h.gc.10"} scheint am wenigsten \textit{Gene mit ES-SE} und \textit{"ES-SE Gesamt"} zu besitzen.
Bei der Maus wiederum gibt es vergleichsweise wenige \textit{ES-SE}, obwohl es insgesamt fast genau so viele
protein-kodierende Gene gibt (23119) wie in \textit{"h.ens.75"} (23393).

Unter Einbezug der Abbildungen \ref{fig:-plots-skipped_bases-jpg} und \ref{fig:-plots-skipped_exons-jpg}, sind die
selben vier Trends zu beobachten, die sich in Abbildung \ref{fig:-plots-genes-jpg} bereits andeuten:

\begin{figure}[htpb]
	\centering
	\includegraphics[width=0.8\textwidth]{./plots/skipped_bases.jpg}
	\caption{Kumulative Verteilung der übersprungenen Basen pro \textit{GTF}}
	\label{fig:-plots-skipped_bases-jpg}
\end{figure}


Abbildung \ref{fig:-plots-skipped_bases-jpg} zeigt die kumulative Verteilung der übersprungenen Basen pro \textit{GTF}.
Alle Kurven zeigen einen charakteristischen S-förmigen Verlauf, was auf ein ähnliches grundlegendes Muster im \textit{ES-SE} Verhalten
hindeutet. Es bilden sich hauptsächlich zwei Plateaus aus:
Ein höheres Plateau bei etwa 15.000-17.500 übersprungenen Basen für die vom Mensch stammenden \textit{GTF} Dateien
und ein niedrigeres Plateau bei etwa 6.000 übersprungenen Basen für die Maus.
Die zwei Ausrei\ss er (\textit{"h.ens.75"} und \textit{"h.gc.10"}) des höheren Plateaus sind wieder auf die jeweils
grö\ss te und kleinste Anzahl an Genen mit \textit{ES-SE} innerhalb der Humanen \textit{GTF} Dateien zurückzuführen.

Der steilste Anstieg der Kurven erfolgt im Bereich zwischen 100 und 1.000 übersprungenen Basen, was darauf hindeutet,
dass die meisten \textit{ES-SE} in diesem Größenbereich stattfinden.
Die logarithmische Skalierung der x-Achse verdeutlicht, dass die \textit{ES-SE} über mehrere Größenordnungen hinweg auftreten,
von einzelnen Basen bis hin zu mehreren tausend Basen.

\begin{figure}[htpb]
	\centering
	\includegraphics[width=0.8\textwidth]{./plots/skipped_exons.jpg}
	\caption{Kumulative Verteilung der übersprungenen Exons pro \textit{GTF}}
	\label{fig:-plots-skipped_exons-jpg}
\end{figure}

Abbildung \ref{fig:-plots-skipped_exons-jpg} zeigt die kumulative Verteilung der übersprungenen Exons pro \textit{GTF}.
Auch hier zeigen sich die 4 Trends aus \ref{fig:-plots-skipped_bases-jpg} und \ref{fig:-plots-genes-jpg}.
Die meisten \textit{ES-SE} betreffen lediglich ein oder zwei Exons und werden mit zunehmender Anzahl an
übersprungenen Exons immer weniger.
Diese Verteilung unterstreicht die biologische Relevanz von \textit{Single-Exon-Skipping} als häufigstem Mechanismus im alternativen
Spleißen und zeigt gleichzeitig, dass komplexere \textit{ES-SE} mit mehreren Exons zwar vorkommen,
aber deutlich seltener sind.

Die Tabellen in der Abbildung \ref{tab:top} listen Gene nach der Anzahl der übersprungenen Basen und Exons auf.
Die Tabellen bestehen hier zu 90\% von Genen aus dem Menschen, lediglich ein Gen der Maus (\textbf{\textit{"Ttn"}}) konkurriert mit den 
anderen Genen der Rangliste. Dabei ist \textit{"TTN"} in beiden Kategorien auf dem ersten Platz mit über 26000 übersprungenen Basen und fast 
170 übersprungenen Exons. \textit{"TTN"} und \textbf{\textit{"Ttn"}} sind eng verwandte Gene, welche beide für das Protein \textit{Titin} in den 
jeweiligen Organismen verantwortlich sind. \textit{Titin} wiederum hat eine sehr wichtige Rolle in der 
Muskelkontraktion und dient zur Stabilisierung und Flexibilität der Sarkomere (\cite{uniprot_titin}). Zudem
sind \textit{"TTN"} und \textbf{\textit{"Ttn"}} jeweils mit 34350 (\cite{uniprot_TTN}) und 35213 (\cite{uniprot_Ttn}) extrem lange Proteine,
was die hohe Position in der Rangliste erklärt.

Aus der unteren Hälfte der Tabelle geht hervor, dass eine hohe Anzahl an übersprungenen Exons nicht unbedingt mit 
der Anzahl an übersprungenen Basen zusammenhängen muss. Während die Gene \textit{"NBPF20"}, \textit{"DYNC2H1"} und \textit{"NBPF10"}
alle mindestens 70 Exons übersprungen haben, werden die gleichen Ränge in der Tabelle zu übersprungenen Basen von anderen 
Genen belegt (\textit{"MUC4"} und \textit{"ADGRV1"}).

\begin{figure}[htbp]
	\centering
	\begin{minipage}{0.45\textwidth}
		\centering
		\begin{tabular}{l|l|l}
			\textit{\textbf{Symbol}} & \textit{\textbf{ID}}                                                                                                                      & \textit{\textbf{Basen}} \\\hline
			TTN                      & \href{https://asia.ensembl.org/Homo_sapiens/Gene/Summary?db=core;g=ENSG00000155657;r=2:178525989-178830802}{ENSG00000155657}              & 26106                   \\
			TTN                      & \href{https://asia.ensembl.org/Homo_sapiens/Gene/Summary?db=core;g=ENSG00000155657;r=2:178525989-178830802}{ENSG00000155657.25}           & 26106                   \\
			\textbf{Ttn}                      & \href{https://asia.ensembl.org/Mus_musculus/Gene/Summary?db=core;g=ENSMUSG00000051747;r=2:76534324-76812891}{\textbf{ENSMUSG00000051747}} & 24843                   \\
			TTN                      & \href{https://asia.ensembl.org/Homo_sapiens/Gene/Idhistory?g=ENSG00000283186}{ENSG00000283186}                                            & 22134                   \\
			TTN                      & \href{https://asia.ensembl.org/Homo_sapiens/Gene/Summary?db=core;g=ENSG00000155657;r=2:178525989-178830802}{ENSG00000155657.16}           & 22134                   \\
			TTN                      & \href{https://asia.ensembl.org/Homo_sapiens/Gene/Idhistory?g=ENSG00000283186}{ENSG00000283186.1}                                          & 22134                   \\
			MUC4                     & \href{https://asia.ensembl.org/Homo_sapiens/Gene/Summary?db=core;g=ENSG00000145113;r=3:195746765-195811973}{ENSG00000145113}              & 12875                   \\
			MUC4                     & \href{https://asia.ensembl.org/Homo_sapiens/Gene/Summary?db=core;g=ENSG00000145113;r=3:195746765-195811973}{ENSG00000145113.16}           & 12875                   \\
			MUC4                     & \href{https://asia.ensembl.org/Homo_sapiens/Gene/Summary?db=core;g=ENSG00000145113;r=3:195746765-195811973}{ENSG00000145113.21}           & 12875                   \\
			ADGRV1                   & \href{https://asia.ensembl.org/Homo_sapiens/Gene/Summary?db=core;g=ENSG00000164199;r=5:90529344-91164437}{ENSG00000164199}                & 12530                   \\
		\end{tabular}
	\end{minipage}%
	\hfill
	\begin{minipage}{0.45\textwidth}
		\centering
        \begin{tabular}{l|l|l}
			\textit{\textbf{Symbol}} & \textit{\textbf{ID}}                                                                                                                      & \textit{\textbf{Exons}} \\\hline
			TTN                      & \href{https://asia.ensembl.org/Homo_sapiens/Gene/Summary?db=core;g=ENSG00000155657;r=2:178525989-178830802}{ENSG00000155657}              & 169                     \\
			TTN                      & \href{https://asia.ensembl.org/Homo_sapiens/Gene/Summary?db=core;g=ENSG00000155657;r=2:178525989-178830802}{ENSG00000155657.25}           & 169                     \\
			\textbf{Ttn}                      & \href{https://asia.ensembl.org/Mus_musculus/Gene/Summary?db=core;g=ENSMUSG00000051747;r=2:76534324-76812891}{\textbf{ENSMUSG00000051747}} & 154                     \\
			TTN                      & \href{https://asia.ensembl.org/Homo_sapiens/Gene/Idhistory?g=ENSG00000283186}{ENSG00000283186}                                            & 121                     \\
			TTN                      & \href{https://asia.ensembl.org/Homo_sapiens/Gene/Summary?db=core;g=ENSG00000155657;r=2:178525989-178830802}{ENSG00000155657.16}           & 121                     \\
			TTN                      & \href{https://asia.ensembl.org/Homo_sapiens/Gene/Idhistory?g=ENSG00000283186}{ENSG00000283186.1}                                          & 121                     \\
			NBPF20                   & \href{https://asia.ensembl.org/Homo_sapiens/Gene/Idhistory?g=ENSG00000203832}{ENSG00000203832.5}                                          & 78                      \\
			DYNC2H1                  & \href{https://asia.ensembl.org/Homo_sapiens/Gene/Summary?db=core;g=ENSG00000187240;r=11:103109410-103479863}{ENSG00000187240}             & 70                      \\
			NBPF10                   & \href{https://asia.ensembl.org/Homo_sapiens/Gene/Summary?db=core;g=ENSG00000271425;r=1:146064711-146229000}{ENSG00000271425}              & 70                      \\
			DYNC2H1                  & \href{https://asia.ensembl.org/Homo_sapiens/Gene/Summary?db=core;g=ENSG00000187240;r=11:103109410-103479863}{ENSG00000187240.8}           & 70                      \\
		\end{tabular}
	\end{minipage}
    \caption{Rangliste an Genen mit höchster Anzahl an übersprungen Basen und Exons}\label{tab:top}
\end{figure}




% ------------------------------------------------------------------------------








% ------------------------------------------------------------------------------
\printbibliography
% ------------------------------------------------------------------------------


% ------------------------------------------------------------------------------
\newpage~\appendix
% ------------------------------------------------------------------------------



\end{document}

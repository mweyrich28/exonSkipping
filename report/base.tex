\documentclass[12pt]{article}
\usepackage{paper,math}
\addbibresource{references.bib}

\title{Genomorientierte Bioinformatik - Report ExonSkipping}
\author{
  Malte Weyrich
}
\date{\today}

\hypersetup{
  pdftitle = {Example Paper Title},
  pdfauthor = {Author One, Author Two},
}

% Conditionally display thoughts (hide by switching to `\boolfalse`)
\boolfalse{INCLUDECOMMENTS}
\newcommand{\kyle}[1]{\coauthorComment[Kyle]{#1}}

\begin{document}

% Title Page -------------------------------------------------------------------
\maketitle
\begin{abstract}
    \textbf{Exon Skipping Splicing Events} (\textit{ES-SE}) beschreiben, wie co- oder posttranslational,
    manche Exons eines Transkripts durch das Splei\ss osom herausgeschnitten oder übersprungen werden, während
    in anderen Transkripten des selben Gens, diese weiterhin Teil der mRNA bleiben.
    Die \textit{ES-SE} lassen sich anhand von \textbf{Gene Transfer Format} (\textit{gtf}) files,
    also Genom Annotations Dateien ablesen und analysieren. 
    Im Folgenden wird ein Programm zur Erkennung von allen \textit{ES-SE} innerhalb eines Genoms
    anhand seiner Logik, Laufzeit und Ergebnisse analysiert, wobei nur \textit{ES-SE} berücksichtigt werden,
    die protein-kodierende Transkripte betreffen. Das Programm wurde auf allen verfügbaren
    \textit{gtf} Dateien in \textit{/mnt/biosoft/praktikum/genprakt/gtfs/} ausgeführt.

\end{abstract}

\newpage


% Paper ------------------------------------------------------------------------

% ------------------------------------------------------------------------------
\section{ES-SE Definition}\label{sec:problem}
In einem Gen kann jedes Transkript jeweils mehrere \textit{ES-SE} haben.
Ein \textit{ES-SE} involviert immer jeweils mindestens eine \textbf{Splice Variant} (\textit{SV}) und einen 
\textbf{Wild Type} (\textit{WT}). Beide dieser Begriffe beziehen sich auf Transkripte eines Gens $G$.
Ein \textit{SV} ist ein Transkript $T_{SV}$, welches ein Intron $I$ mit Startposition $I_{S}$ und Endposition
$I_{E}$ besitzt, was gleichzeitig bedeutet, dass es in $T_{SV}$ zwei Exons $A, B$ gibt, die $I$ flankieren. Somit endet $A$ bei $I_{S} - 1 = A_{E}$ und $B$ startet bei $I_{E} + 1 = B_{S}$.
Zudem ist die Position $B_{pos} - A_{pos} = 1$, wobei sich $A_{pos}$ auf die Position von Exon $A$ relativ gesehen 
zu allen anderen Exons von $T_{SV}$ bezieht.
Ein \textit{WT} wäre nun ein weiteres Transkript $T_{WT}$ des selben Gens $G$, welches ebenfalls
zwei Exons $C, D$ besitzt mit $C_{pos} < D_{pos}$, wobei $C_{E} = I_{S} - 1$ und $D_{S} = I_{E} + 1$, 
jedoch gilt für $C, D$: $D_{pos} - C_{pos} > 1$. 
Dies bedeutet, dass die Exons von $T_{WT}$ zwischen $C$ und $D$ in $T_{SV}$ herausgesplei\ss t wurden.
Es kann pro Event mehrere \textit{SV}'s und \textit{WT}'s geben.

\section{Java Programm}
\subsection{Logik}\label{sec:logik}
Der Workflow der \textit{JAR} lässt sich in drei Schritte Aufteilen:
\begin{enumerate}
    \item[I] \textbf{Einlesen der \textit{gtf} Datei und Initialisierung der Datenstruktur}

        Zum einlesen wird die \textit{gtf} Datei zuerst nach relevanten Zeilen gefiltert, denn
        für uns sind momentan nur Zeilen relevant, die in der 3. Spalte entweder
        \textit{"exon"} oder \textit{"CDS"} stehen haben. Hierbei wird vermieden,
        die Methode \textit{String.split("\textbackslash t")} zu verwenden.
        Stattdessen wird in einem \textit{for loop} jedes Zeichen einzeln betrachtet.
        Dabei werden Zeilen die mit einem \textit{"\#"} anfangen direkt übersprungen.
        Für alle anderen Zeilen werden die Anzahl der \textit{tabs} gezählt und nach 
        dem zweiten \textit{tab}, werden alle darauf folgenden Zeichen zu einem 
        \textit{String} zusammen konkateniert, bis der dritte \textit{tab} 
        erreicht wurde. Falls der entstandenen String $\in \left\{\text{"exon"}, \text{"CDS"} \right\}$, wird die 
        Zeile einer \textit{ArrayList<String>} hinzugefügt, ansonsten wird mit der nächsten Zeile weiter gemacht. 
        Diese Liste enthält am Ende alle relevanten Zeilen. 
        Jede der relevanten Zeilen werden nun mit \textit{String.split("\textbackslash t")} in
        ein \textit{String[] mainComponents} geschrieben. Die \textit{attributeColumn} wird aus
        \textit{mainComponents} extrahiert (auch als ein \textit{String[]} Names \textit{attributes}),
        indem man \textit{mainComponents[mainComponents.length - 1]} am \textit{";"} splitted.

        Mit diesen zwei Komponenten pro Zeile wird als erstes die \textit{gene\_id} abgespeichert
        und überprüft, ob wir eine neue \textit{gene\_id} erreicht haben. 
        Falls ja, wird ein neues Gen erstellt. Für die darauf folgenden Zeilen wird überprüft,
        ob wir ein neues Transkript erreicht haben. Neue Transkripte werden in einer
        \textit{ArrayList<Transkript>} des dazugehörigen Gens abgespeichert. 
        Transkripte wiederum besitzen eine \textit{ArrayList<CodingDnaSequence> cdsList} und zwei
        \textit{HashMap<Integer, CodingDnaSequence>} \textit{cdsStartIndices}, \textit{cdsEndIndices}. Die Transkripte werden mit den dazugehörigen
        \textit{CodingDnaSequence}'s befüllt, wobei für jede erstellte \textit{CodingDnaSequence} die Start- und 
        Endposition in den jeweiligen \textit{HashMap}'s als Key auf das erstellte Objekt verweisen.
        Zudem wird mit einer Zählvariable \textit{int cdsCount} die Position der \textit{CodingDnaSequence}'s
        innerhalb des Transkripts in dem \textit{CodingDnaSequence} Objekt gespeichert.

    \item[II] \textbf{Generieren der \textit{ES-SE}}

        Zum generieren der \textit{ES-SE} werden als erstes für alle in dem Genom abgespeicherten
        Gene, die dazugehörigen \textit{Introns} errechnet und in einem \textit{HashSet<Introns>} 
        innerhalb des Gens abgespeichert. Dafür werden alle Transkripte eines Gens und deren
        \textit{CodingDnaSequence}'s angeschaut. Die Introns werden dann mit jeweils
        zwei \textit{CodingDnaSequence}'s berechnet
        (bei Genen die sich auf dem \textit{"-"} Strang befinden, müssen zuerst die 
        \textit{cdsList}'s aller Transkripte invertiert werden und die 
        Positionen der \textit{CodingDnaSequence}'s neu berechnet werden.
        Das ist später relevant für die Identifikation der \textit{WT}'s):
    \begin{verbatim}

// invert cdsList of transcripts
public void invertTranscripts() {
    for (int i = 0; i < transcripts.size(); i++) {
        Transcript currTranscript = transcripts.get(i);
        currTranscript.reversCdsList();
        // updating pos attribute of each cds
        for (int j = 0; j < currTranscript.getCdsList().size(); j++) {
            currTranscript.getCdsList().get(j).setPos(j);
        }
    }
}

// generating introns
for (Transcript transcript : transcripts) {
    for (int i = 0; i < transcript.getCdsList().size() - 1; i++) {
        int intronStart = transcript.getCdsList().get(i).getEnd() + 1;
        int intronEnd = transcript.getCdsList().get(i + 1).getStart() - 1;
        Intron intron = new Intron(intronStart, intronEnd);
        introns.add(intron);
    }
}
    \end{verbatim}
    
    Anschlie\ss end wird für jedes Gen $G$ über die Intron Liste iteriert. Für jedes Intron $I$
    müssen alle Transkripte von $G$ nach \textit{CodingDnaSequence}'s $A, B$ durchsucht werden, die
    die Bedingung $A_{E} + 1 = I_{S}$ und $B_{S} - 1 = I_{E}$. Dies kann mit Hilfe der zwei 
    \textit{HashMap<Integer, CodingDnaSequence>} Objekte durchgeführt werden.
    Zudem wir für jedes Intron $I$ jeweils 4 leere \textit{HashSet<String>}'s erstellt:
    \begin{enumerate}
        \item[1.] \textit{SV\_INTORN}: enthält \textit{"intronStart:intronEnd"} des momentanen Introns $I$
        \item[2.] \textit{SV\_PROTS}: enthält die \textit{proteinId} von \textit{CodingDnaSequence} $A$
            \item[3.] \textit{WT\_INTORN}: enthält alle \textit{"intronStart:intronEnd"} Koordinaten, die zwischen $A$ und $B$ liegen
        \item[4.] \textit{WT\_PROTS}: enthält die \textit{proteinId}'s von allen \textit{CodingDnaSequence}'s die zwischen $A$ und $B$ liegen
    \end{enumerate}
    Nun gibt es zwei Möglichkeiten:
    \begin{enumerate}
        \item  $A_{E} + 1 = I_{S}$ und $B_{S} - 1 = I_{E}$ und $B_{pos} - A_{pos} = 1$
        \item  $A_{E} + 1 = I_{S}$ und $B_{S} - 1 = I_{E}$ und $B_{pos} - A_{pos} > 1$
    \end{enumerate}

    Falls $i$ eintrifft, handelt es sich um ein \textit{SV} und es wird die \textit{proteinId} von \textit{CodingDnaSequence} $A$
    in \textit{SV\_PROTS} aufgenommen.
    Ansonsten werden bei Fall $ii$ alle \textit{proteinId}'s der \textit{CodingDnaSequence}'s zwischen $A$ und $B$ zu 
    \textit{WT\_PROTS} und alle Introns zwischen $A$ und $B$ zu \textit{WT\_INTORN} hinzugefügt.
    Dabei werden ebenfalls Werte wie \textit{min/max\_skipped\_exon/bases} berechnet:
    \begin{verbatim}
// add all introns of WT to WT_INTRON and all cdsids/prot_ids to WT_prots
int skippedBases = 0;
for (int i = cdsFront.getPos() ; i < cdsBehind.getPos(); i++) {
    int wtIntronStart = cdsList.get(i).getEnd() + 1;
    int wtIntronEnd = cdsList.get(i+1).getStart();
    WT_INTRON.add(wtIntronStart + ":" + wtIntronEnd);

    // like this i add many ids twice but that's fine :)
    WT_PROTS.add(cdsFront.getId());
    WT_PROTS.add(cdsBehind.getId());

    if (i > cdsFront.getPos() && i < cdsBehind.getPos()) {
        // we are in a cds that was skipped
        // → get end - start + 1 = length → add to skipped bases
        skippedBases += cdsList.get(i).getEnd() 
                        - cdsList.get(i).getStart() + 1;
    }
}
    \end{verbatim}
    Ein \textit{ES-SE} wird nur in die \textit{ArrayList<String> events} aufgenommen, falls es für das momentane
    Intron $I$ mindestens einen \textit{WT} gab.

    \item[III] \textbf{Erstellen der \textit{<out>.tsv} Datei}

    Die \textit{ArrayList<String> events} enthält nun alle \textit{ES-SE} als \textit{String} in bereits
    korrekter Formatierung. In einem \textit{for loop} wird die Lösung Zeile für Zeile in ein \textit{out.tsv}
    geschrieben.
\end{enumerate}
\newpage
\subsection{Laufzeit}
Die Laufzeitanalyse wird in die drei Segmente aus \ref{sec:logik} unterteilt.
\begin{enumerate}
    \item[I] \textbf{Einlesen der \textit{gtf} Datei und Initialisierung der Datenstruktur}

        Für eine \textit{gtf} Datei mit $m$ Zeilen benötigt die Selektion der relevanten Zeilen
        schon mal mindestens $m$ Vergleiche, da jede Zeile überprüft werden muss.
        Für jede Zeile wird ein Substring ab dem zweiten \textit{Tab} bis zum dritten \textit{Tab} erstellt (au\ss er bei Kommentaren, diese werden übersprungen). 
        Die Anzahl der Vergleiche pro Zeile ist kleiner als die Länge der Zeile (da wir ab dem dritten \textit{Tab} abbrechen) und lässt 
        sich als $a < m.length$ beschreiben. Also
        \begin{equation}
            \mathcal{O}(m \cdot a) \implies \mathcal{O}(m)
        \end{equation}
        ,da $a$ eine Konstante ist.
        
        Bei einer \textit{gtf} Datei mit $m$ validen Zeilen (d.h. jede Zeile hat entweder einen \textit{"exon"} oder \textit{"CDS"} Eintrag) bleiben
        nach dem Filtern $m$ Zeilen übrig.
        Für jede dieser Zeilen muss:
        \begin{enumerate}
            %% https://softwareengineering.stackexchange.com/questions/331909/whats-the-complexity-of-javas-string-split-function
            \item Die Zeile am \textit{Tab} geteilt werden: 
                \begin{equation}
                    \mathcal{O}(n)
                \end{equation}
                , wobei $n$ die Länge der Zeile ist.
            \item Die letzte Komponente aus i. am \textit{;} geteilt werden: lässt sich ebenfalls mit 
                \begin{equation}
                    \mathcal{O}(n)
                \end{equation}
                von oben beschränken.
            \item Die \textit{gene\_id} aus den Attributen aus ii. mit \textit{String parseAttributes(String[] attributeEntries, String attributeName)}
                abfragen, also:
                \begin{equation}
                   \mathcal{O}(e \cdot e_{l})
                \end{equation}
                , wobei $e$ die Länge des \textit{attributeEntries} Arrays ist und
                $e_{l}$ die Länge des längsten Eintrags in \textit{attributeEntries} ist, da wir im Worst Case
                über alle Einträge in \textit{attributeEntries} iterieren ($e$) und für jeweils jeden Eintrag
                mindesten vier String Operationen (\textit{trim()}, \textit{indexOf()}, zwei mal \textit{substring()})
                und eine Vergleichsoperation (\textit{equals()}) aufrufen, welche maximal $e_{l}$ viele 
                Operationen benötigen. Also $\mathcal{O}(e \cdot 5 \cdot e_{l}) = \mathcal{O}(e \cdot e_{l})$.
                Für jedes weitere Vorkommen von \textit{parseAttributes()} wird diese Komplexität angenommen.
            \item Ein neues Gen initialisiert werden und der \textit{gene\_name} abgefragt werden (falls ein neues Gen erreicht wurde), also: 
                \begin{equation}
                    \mathcal{O}(1 + e \cdot e_{l})   
                \end{equation}
            \item Die \textit{transcript\_id} abgefragt werden, also
                \begin{equation}
                    \mathcal{O}(e \cdot e_{l})
                \end{equation}
            \item Falls es sich um einen \textit{"CDS"} Eintrag handelt, muss die \textit{protein\_id} abgefragt werden, also:
                \begin{equation}
                    \mathcal{O}(e \cdot e_{l})   
                \end{equation}
                und in Konstanter Zeit ggf. neue Objekte erstellt, oder auf bereits existierende Objekte
                zugreifen, um ein neues \textit{CodingDnaSequence} Objekt zu erstellen.
        \end{enumerate}

        Insgesamt hat die Einlese Routine also eine Komplexität von
        \[
            (A) \hspace{3mm} \mathcal{O}(m \cdot 2 \cdot n \cdot 4 \cdot  (e \cdot e_{l})) = \mathcal{O}(m \cdot n \cdot e \cdot e_{l})
        .\]
    \item[II] \textbf{Generieren der \textit{ES-SE}}

    Sei $g$ die Anzahl an Genen in unserem Genom. Da wir vom Worst Case ausgehen sagen wir, 
    dass sich jedes Gen auf dem \textit{"-"} Strang befindet.
    So muss zuerst in jedem Transkript jedes Gens die \textit{cdsList} invertiert werden.
    Dies geschieht in:
    \begin{equation}
        \mathcal{O}(g \cdot g_{t} \cdot 2 \cdot t_{c}) = \mathcal{O}(g \cdot g_{t} \cdot t_{c})
    \end{equation}
    , wobei $g_{t}$ die grö\ss te Anzahl an Transkripten von $g$ ist und $t_{c}$ die längste \textit{cdsList} eines Transkripts ist.
    Die Konstante $2$ kommt zustande, da zuerst die \textit{cdsList} mit \textit{Collections.reverse()} umgekehrt wird($\mathcal{O}(g_{t})$) und
    dann nochmals in $\mathcal{O}(g_{t})$ durchlaufen wird, um die in den \textit{CodingDnaSequence} Objekten gespeicherte Position
    anzupassen.
    Dann werden für jedes Gen $g$ die Introns generiert. Dies geschieht ebenfalls in
    \begin{equation}
        \mathcal{O}(g \cdot g_{t} \cdot t_{c})
    \end{equation}
    Die \textit{ES-SE} werden in der \textit{getEvents()} Methode berechnet.
    Dafür müssen für alle Gene alle Introns und alle Transkripte überprüft werden.
    Also schon mal $\mathcal{O}(g \cdot g_{i} \cdot g_{t})$.
    Hier ist $g_{i}$ die grö\ss te Anzahl an Introns von allen Genen und $g_{t}$ wieder 
    die grö\ss te Anzahl an Transkripten aller Gene. Alle anderen Operationen sind
    konstant in ihrer Komplexität, da bei ihnen lediglich bereits existente Werte
    in \textit{HashMaps} oder Objekten abgefragt werden. 
    Nur falls ein \textit{WT} entdeckt wird, wird in einem \textit{for loop} über die 
    \textit{CodingDnaSequence}'s zwischen $C$ und $D$ (siehe \ref{sec:problem}. ES-SE Definition) iteriert.
    Sei $mSE$ (= maxSkippedExons) also die von allen \textit{ES-SE} eines Genoms
    maximale Anzahl an übersprungenen Exons, so wäre die gesamte Komplexität von
    II:
    \begin{equation}
    (B) \hspace{3mm} \mathcal{O}(2 \cdot (g \cdot g_{t} \cdot t_{c}) + g \cdot g_{i} \cdot g_{t} \cdot mSE) = \mathcal{O}(\overbrace{g \cdot g_{t} \cdot t_{c}}^{\shortstack{\scriptsize\text{invertTranscripts()} \\ \text{\scriptsize\& generateIntrons()}}} + \underbrace{g \cdot g_{i} \cdot g_{t} \cdot mSE}_{\text{getEvents()}} )
    \end{equation}

    \item[III] \textbf{Erstellen der \textit{<out>.tsv} Datei}

    Hat eine Komplexität von 
    \begin{equation}
        (C) \hspace{3mm} \mathcal{O}(E)
    \end{equation}
    , wenn $E$ die Menge aller \textit{ES-SE} ist.
\end{enumerate}

Zusammenfassend also eine Gesamtkomplexität von:
\begin{align*}
    (A) + (B) + (C) = \mathcal{O}( m \cdot n \cdot e \cdot e_{l}+ g \cdot g_{t} \cdot t_{c} + g \cdot g_{i} \cdot g_{t} \cdot mSE + E)
\end{align*}
    


\section{Ergebnisse}
% ------------------------------------------------------------------------------








% ------------------------------------------------------------------------------
% \printbibliography
% ------------------------------------------------------------------------------


% ------------------------------------------------------------------------------
\newpage~\appendix
% ------------------------------------------------------------------------------

\section{Appendix Section}

hm 

Text goes here



\end{document}

\documentclass[12pt]{article}
\usepackage{paper,math}
\addbibresource{references.bib}

\title{Genomorientierte Bioinformatik - Report ExonSkipping}
\author{
  Malte Weyrich
}
\date{\today}

\hypersetup{
  pdftitle = {Example Paper Title},
  pdfauthor = {Author One, Author Two},
}

% Conditionally display thoughts (hide by switching to `\boolfalse`)
\boolfalse{INCLUDECOMMENTS}
\newcommand{\kyle}[1]{\coauthorComment[Kyle]{#1}}

\begin{document}

% Title Page -------------------------------------------------------------------
\maketitle
\begin{abstract}
    \textbf{Exon Skipping Splicing Events} (\textit{ES-SE}) beschreiben, wie co- oder posttranslational
    manche Exons eines Transkripts durch das Splei\ss osom herausgeschnitten oder übersprungen werden, während
    in anderen Transkripten des selben Gens, diese weiterhin Teil der mRNA bleiben.
    Die \textit{ES-SE} lassen sich anhand von \textbf{Gene Transfer Format} (\textit{gtf}) files,
    also Genom Annotations Dateien ablesen und analysieren. 
    Im Folgenden wird ein Programm zur Erkennung von allen \textit{ES-SE} innerhalb eines Genoms
    anhand seiner Logik, Laufzeit und Ergebnisse Analysiert, wobei nur \textit{ES-SE} berücksichtigt werden,
    die protein-kodierende Transkripte betreffen. Das Programm wurde auf allen verfügbaren
    \textit{gtf} Dateien in $/mnt/biosoft/praktikum/genprakt/gtfs/$ ausgeführt.

\end{abstract}

\newpage


% Paper ------------------------------------------------------------------------

% ------------------------------------------------------------------------------
\section{ES-SE Definition}
In einem Gen kann jedes Transkript jeweils mehrere \textit{ES-SE} haben.
Ein \textit{ES-SE} involviert immer jeweils mindestens eine \textbf{Splice Variant} (\textit{SV}) und einen 
\textbf{Wild Type} (\textit{WT}). Beide dieser Begriffe beziehen sich auf Transkripte eines Gens $G$.
Ein \textit{SV} ist ein Transkript $T_{SV}$, welches ein Intron $I$ mit Startposition $I_{S}$ und Endposition
$I_{E}$ besitzt, was gleichzeitig bedeutet, dass es in $T_{SV}$ zwei Exons $A, B$ gibt,
die $I$ flankieren. Somit endet $A$ bei $I_{S} - 1 = A_{E}$ und $B$ startet bei $I_{E} + 1 = B_{S}$.
Zudem ist die Position $B_{pos} - A_{pos} = 1$, wobei sich $A_{pos}$ auf die Position von Exon $A$ relativ gesehen 
zu allen anderen Exons von $T_{SV}$ bezieht.
Ein \textit{WT} wäre nun ein weiteres Transkript $T_{WT}$ des selben Gens $G$, welches ebenfalls
zwei Exons $C, D$ besitzt mit $C_{pos} < D_{pos}$, wobei $C_{E} = I_{S} - 1$ und $D_{S} = I_{E} + 1$, 
jedoch gilt für $C, D$: $D_{pos} - C_{pos} \ge 1$. 
Dies bedeutet, dass die Exons von $T_{WT}$ zwischen $C$ und $D$ in $T_{SV}$ herausgesplei\ss t wurden.

\section{Java Programm}
\subsection{Logik}
Der Workflow der \textit{JAR} lässt sich in drei Schritte Aufteilen:
\begin{enumerate}
    \item[I] \textbf{Einlesen der \textit{gtf} Datei und Initialisierung der Datenstruktur:}

        Zum einlesen wird die \textit{gtf} Datei zuerst nach relevanten Zeilen gefiltert, denn
        für uns sind momentan nur Zeilen relevant, die in der 3. Spalte entweder
        \textit{"exon"} oder \textit{"CDS"} stehen haben. Hierbei wird vermieden,
        die Methode \textit{String.split("\textbackslash t")} zu verwenden.
        Stattdessen wird in einem \textit{for loop} jedes Zeichen einzeln betrachtet.
        Dabei werden Zeilen die mit einem \textit{"\#"} anfangen direkt übersprungen.
        Für alle anderen Zeilen werden die Anzahl der \textit{tabs} gezählt und nach 
        dem zweiten \textit{tab}, werden alle darauf folgenden Zeichen zu einem 
        \textit{String} zusammen konkateniert, bis der dritte \textit{tab} 
        erreicht wurde. Falls der entstandenen String $\in \left\{\text{"exon"}, \text{"CDS"} \right\}$, wird die 
        Zeile einer \textit{ArrayList<String>} hinzugefügt, ansonsten wird mit der nächsten Zeile weiter gemacht. 
        Diese Liste enthält am Ende alle relevanten Zeilen. 
        Jede der relevanten Zeilen werden nun mit \textit{String.split("\textbackslash t")} in
        ein \textit{String[] mainComponents} geschrieben. Die \textit{attributeColumn} wird aus
        \textit{mainComponents} extrahiert (auch als ein \textit{String[]} Names \textit{attributes}),
        indem man \textit{mainComponents[mainComponents.length - 1]} am \textit{";"} splitted.

        Mit diesen zwei Komponenten pro Zeile wird als erstes die \textit{gene\_id} abgespeichert
        und überprüft, ob wir eine neue \textit{gene\_id} erreicht haben. 
        Falls ja, wird ein neues Gen erstellt. Für die darauf folgenden Zeilen wird überprüft,
        ob wir ein neues Transkript erreicht haben. Neue Transkripte werden in einer
        \textit{ArrayList<Transkript>} des dazugehörigen Gens abgespeichert. 
        Transkripte wiederum besitzen eine \textit{ArrayList<CodingDnaSequence> cdsList} und zwei
        \textit{HashMap<Integer, CodingDnaSequence>} \textit{cdsStartIndices}, \textit{cdsEndIndices}. Die Transkripte werden mit den dazugehörigen
        \textit{CodingDnaSequence}'s befüllt, wobei für jede erstellte \textit{CodingDnaSequence} die Start- und 
        Endposition in den jeweiligen \textit{HashMap}'s als Key auf das erstellte Objekt verweisen.
        Zudem wird mit einer Zählvariable \textit{int cdsCount} die Position der \textit{CodingDnaSequence}'s
        innerhalb des Transkripts in dem \textit{CodingDnaSequence} Objekt gespeichert.

    \item[II] \textbf{Generieren der \textit{ES-SE}}

        Zum generieren der \textit{ES-SE} werden als erstes für alle in dem Genom abgespeicherten
        Gene, die dazugehörigen \textit{Introns} errechnet und in einem \textit{HashSet<Introns>} 
        innerhalb des Gens abgespeichert. Dafür werden alle Transkripte eines Gens und deren
        \textit{CodingDnaSequence}'s angeschaut. Die Introns werden dann mit jeweils
        zwei \textit{CodingDnaSequence}'s berechnet
        (Bei Genen die sich auf dem \textit{"-"} Strang befinden, müssen zuerst die 
        die \textit{cdsList}'s aller Transkripte invertiert werden und die 
        Positionen der \textit{CodingDnaSequence}'s neu berechnet werden.
        Das ist später relevant für die Identifikation der \textit{WT}'s):
    \begin{verbatim}
for (Transcript transcript : transcripts) {
    for (int i = 0; i < transcript.getCdsList().size() - 1; i++) {
        int intronStart = transcript.getCdsList().get(i).getEnd() + 1;
        int intronEnd = transcript.getCdsList().get(i + 1).getStart() - 1;
        Intron intron = new Intron(intronStart, intronEnd);
        introns.add(intron);
    }
}
    \end{verbatim}
    

    \item[III] \textbf{Erstellen der \textit{<out>.tsv} Datei}
\end{enumerate}
\subsection{Laufzeit}
\section{Ergebnisse}
% ------------------------------------------------------------------------------








% ------------------------------------------------------------------------------
% \printbibliography
% ------------------------------------------------------------------------------


% ------------------------------------------------------------------------------
\newpage~\appendix
% ------------------------------------------------------------------------------

\section{Appendix Section}

hm 

Text goes here



\end{document}
